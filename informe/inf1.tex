\documentclass[12pt]{article}
\usepackage{enumerate}
\usepackage{graphicx}
\usepackage{caption}
\usepackage{subcaption}
\usepackage{float}
\usepackage{anysize}
\marginsize{1.5cm}{1.5cm}{1.5cm}{1.5cm}

\begin{document}

\title{Informe 1}
 
\author{Grupo  \\Alvarez Vico Jazm\'in\\Cortés Conde Titó Javier María\\Pedraza Marcelo \\ Rozenber Uriel Jonathan}


\maketitle

\clearpage

\tableofcontents
\cleardoublepage

\section{Problema 1: Cruzando el puente}

\subsection{Inrtoducción}

En este problema, un Grupo formado por arqueologos y caníbales debe cruzar un puente. El mismo solo puede ser atravezado por dos personas a la vez, y como el grupo solo dispone de una linterna, siempre debe volver alguien con la linterna. Además en ningun lugar puede haber mas canibales que arqueologos.
Cada individuo consta de la propiedad "velocidad" un numero natural que indica cuanto tarda en atravezar el puente. Sí dos personas atraviezan el puente juntas, lo hacen a la velocidad minima de los dos.
%terminar%

\section{Problema 2: Problemas en el camino}

\subsection{Introducción}

En el problema enunciado, Nuestros exploradores se encuentran con una balanza de dos platos, y en el de la izquierda la llave que necesitan. Para que esta sea de utilidad, al tomarla deben mantener la posición de la balanza y para realizar esto cuentan con pesas de peso igual a las potencias de tres (una pesa por potencia).
Entonces, sabiendo el peso de la llave, necesitamos saber que pesas poner en cada plato para mantener el equilibrio original.
Podemos pensar que el lado derecho de la balanza equivale a la operación de sustracción y el izquiero a la adición. De esta forma al agregar pesas de un lado o del otro estariamos sumando o restando su peso
Formalmente esto equivale a decir que, si tenemos un entero n queremos sumar o restar potencias de tres, sin repetirlas hasta alcanzar su valor.    
 

\section{Problema 3: Guardando el tesoro}

\subsection{Introducción}

En este problema, los exploradores se encuentran frente a muchos tesoros que desearían poder llevarse. Para esto, cuentan con varias mochilas, cada una con una determinada capacidad de peso que puede cargar.
Los tesoros son de distintos tipos, y cada tipo tiene un peso y un valor determinado.
El objetivo es encontrar la manera optima de llenar las mochilas para poder llevar el mayor valor posibe.

Formalmente, tenemos un conjunto de elementos que tienen como propiedad dos naturales asociados (el peso y el valor). Entonces podemos inferir que existen dos criterios de ordenamiento asociados respectivamente a estos valores. 
Al mismo tiempo tenemos otro conjunto de elementos que posee como propiedad un natural asociado (capacidad).
Nuestro objetivo es seleccionar una combinación de los primeros objetos, restringida por las capacidades dadas,de modo de maximizar la sumatoria del valor de los mismos.

% ¿observar sobre permutaciones de mochilas?%
   


\end{document}
