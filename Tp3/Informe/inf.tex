\documentclass[spanish,12pt]{article}
\usepackage[spanish]{babel}
\usepackage[utf8]{inputenc}
\usepackage{xspace}
\usepackage{lmodern}
\usepackage{indentfirst}
\usepackage{xargs}
\usepackage{ifthen}
\usepackage{fancyhdr}
\usepackage{latexsym}
\usepackage{lastpage}
\usepackage{textcomp}
\usepackage{varwidth}
\usepackage{caratula, aed2-tad,aed2-symb,aed2-itef}
\usepackage{algorithmicx, algpseudocode, algorithm}
\usepackage{enumerate}
\usepackage{graphicx}
\usepackage{caption}
\usepackage{subcaption}
\usepackage{float}
\usepackage{anysize}
\marginsize{1.5cm}{1.5cm}{1.5cm}{1.5cm}

\begin{document}

\titulo{Informe 3}
\materia{Algoritmos y Estructuras de Datos III}
\author{Grupo  \\Alvarez Vico Jazm\'in\\Cortés Conde Titó Javier María\\Pedraza Marcelo \\ Rozenberg Uriel Jonathan}

\integrante {Jazmín Alvazer Vico}{75/15}{jazminalvarezvico@gmail.com}
\integrante {Marcelo Pedraza}{393/14}{marcelopedraza314@gmail.com}
\integrante {Uriel Jonathan Rozenberg}{838/12}{rozenberguriel@gmail.com}
\integrante {Javier María Cortés Conde Titó}{252/15}{javiercortescondetito@gmail.com}

\maketitle


\clearpage

\tableofcontents
\cleardoublepage


\section{Introducción}
En nuestro problema Brian quiere convertirse en "maestro pokemon" en el menor tiempo posible. Para lograr este objetivo debe ir a todos los "gimnasios" y conquistarlos. Para poder hacerlo, cada gimnasio requiere una cantidad determinada de pociones. Estas pociones pueden obtenerse en las "pokeparadas". Las pokeparadas solo pueden visitarse una vez y de cada visita obtenemos tres pociones.

Formalmente, podemos caracterizar nuestras pokeparadas y  gimnasios como nodos formando un grafo completo, es decir que existen aristas para unir cualquier par de nodo. nuestras aristas deben tener peso, que equivalga a la distancia entre dos nodos. entonces queremos encontrar el camino mínimo que une todo los nodos gimnasios y los nodos pokeparada que hagan falta para poder conquistar todos los gimnasios. 

%%%%%%%%%%%%%%%%%%%%%%%%%%%%%%%%%%%%

\section{Algoritmo exacto}

\subsection{Exlicación de la solución}

Para modelar este problema, creamos dos clases: Nodo y Mochila.
La clase Mochila contiene dos enteros $``capacidad"$ el cual corresponde con la capacidad de la entrada (k) y $``peso"$ en el cual se guarda la cantidad de pociones que se tiene en cada momento. 
en el Pseudocogigo se llaman a las funciones DameCapacidad y DamePeso, las mismas devuelven estos valores.

La clase Nodo tiene  un entero $``indice"$ como identificador de cada nodoy un entero $``pociones"$ en el cual se guardan las pociones que dan las pokeparadas o las que requieren los gimnasios(en este caso el valor es negativo). Ademas cuenta con un booleano para marcar si esta o no recorrido y dos enteros $``x"$ e $``y"$ para identificar las coordenadas del nodo.

Nuestra solución es utiliza la técnica de Backtracking para resolver el problema. Utilizamos variables globales (las cuales estan mencionadas en el pseudocodigo) ya que las utilizamos todo el tiempo en las funciones de nuestra resolución. De esta forma evitamos pasarlas cada vez como parametros de entrada.

Nuestro algoritmo de BT va formando recursivamente todas las soluciones posibles eligiendo gimnasios y pokeparas siempre que esto sea posible. El ciclo principal de la función va desde 0 hasta el Maximo entre el largo de los vectores de Nodos y pokeparadas. De esta forma vamos probando empezar con todos los nodos posibles. Cada vez que se elige un nodo para continuar se vuelve a llamar a la función BT para seguir construyendo la solución, luego se revierten los cambios para probar las demás soluciones. 

En nuestro algoritmo, cortamos una posible solución en el momento que supera la solución minima encontrada hasta el momento. De esta forma evitamos seguir ejecutando el algoritmo para una solución que claramente no va a ser la mejor, y así reducimos el tiempo de cómputo.

%falta cuandod devuelve -1%

\subsection{Pseudocodigo}

\begin{algorithm}[H]{\textbf{variables Globales}}
	\begin{algorithmic}[1]
		\State real MinActual 
		\State real MinGlobal
		\State vector$<$ int $>$ RecorrifoGlobal
		\State vector$<$ int $>$ RecorridoActual
		\State vector$<$ Nodo $>$ PokeParadas
		\State vector$<$ Nodo $>$ Gimnasios
		\State int PokeParadasRecorridas
		\State int PocionesNecesarias		
		\State Mochila Moch		
	
	\end{algorithmic}
\end{algorithm}

\begin{algorithm}[H]{\textbf{BT}()}
	\begin{algorithmic}[1]
		\State MinActual $\gets \infty$
		\State MinGlobal $\gets 0$
		\State Si  MinActual$>$MinGlobal  
			 \State \quad cortar
		\State finSi
		\State Si GimRecorridos = $|$Gimnasios$|$ % ver donde declaro  GimRecorridos%
		\State \quad Si MinActual $<$ Min Gobal
		\State \qquad MinGlobal $\gets$ MinActual
		\State \qquad RecorridoGlobal $\gets$ RecorridoAtual
		\State \quad finSi
		\State finSi
		\State Desde i=0 hasta i$<$Max($|$Gimnasios$|$,$|$PokeParadas$|$)
	
		\State \qquad Si i$< |$Gimnasios$|$
		\State \qquad \quad gim $\gets$ Gimnasios[i]
		\State \qquad \quad Si puedoIr(gim)
		\State \qquad \qquad Voy(gim)
 		\State \qquad finSi	
		\State \quad finSi
		\State \quad Si i $< |$PokeParadas$|$
		\State \quad pp $\gets$ Pokeparadas[i]
		\State \quad PokeParadasRecorridas $\gets$ PokeParadasRecorridas + 1 
		\State \qquad Si puedoIr(pp)
		\State \qquad Voy(pp) 	
		\State \qquad finSi
		\State \quad PokeParadasRecorridas $\gets$ PokeParadasRecorridas - 1 

		\State \quad finSi 
		\State finDesde
  		
		
		
		\State finSi
	\end{algorithmic}
\end{algorithm}

\begin{algorithm}[H]{\textbf{Voy}(Nodo n)}
	\begin{algorithmic}[1]
		\State Marco n
		\State Si RecorridoActual $ \not= \emptyset$
		\State \quad origen $\gets$ ultimo de RecorridoActual
		\State \quad minActual $\gets$ minActual+ Distancia(origen, p)
		\State finSi
		\State Agrego n a RecorridoActual
		\State Modifico el peso de moch segun las pociones n
		\State BT()
		\State revierto todas las modificaciones
	\end{algorithmic}
\end{algorithm}

\begin{algorithm}[H]{\textbf{puedoIrPP}(Nodo n){bool}}
	\begin{algorithmic}[1]
	\State NoConsumoDeMas $\gets$ Si al entrar a esta pokeparada se tiene que descartar pokebolas que no nos dejan ganar a todos los gimnasios asigno True, si no False.
	%terminar%
	\State  res $\gets$  moch no esta llena y n no esta recorrido y NoConsumoDeMas
	\end{algorithmic}
\end{algorithm}

\begin{algorithm}[H]{\textbf{PuedoIrGim}(Nodo n)}
	\begin{algorithmic}[1]
		\state res $\gets$  n no esta recorrido y DamePeso(moch) $\geq$ -(DamePosciones(n))
	\end{algorithmic}
\end{algorithm}


\subsection{Complejidad}

\subsection{Experimentación y análisis de resultados}

%%%%%%%%%%%%%%%%%%%%%%%%%%%%%%%%%%%%%


\section{Heurística constructiva golosa}

\subsection{Exlicación de la solución}

\subsection{Pseudocodigo}

\subsection{Complejidad}

\subsection{Experimentación y análisis de resultados}


%%%%%%%%%%%%%%%%%%%%%%%%%%%%%%%%%%%%%


\section{Heurística de búsqueda local}

\subsection{Exlicación de la solución}

\subsection{Pseudocodigo}

\subsection{Complejidad}

\subsection{Experimentación y análisis de resultados}

%%%%%%%%%%%%%%%%%%%%%%%%%%%%%%%%%%


\section{Metaheurística}

\subsection{Exlicación de la solución}

\subsection{Pseudocodigo}

\subsection{Complejidad}

\subsection{Experimentación y análisis de resultados}
\end{document}
