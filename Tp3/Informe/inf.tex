\documentclass[spanish,12pt]{article}
\usepackage[spanish]{babel}
\usepackage[utf8]{inputenc}
\usepackage{xspace}
\usepackage{lmodern}
\usepackage{indentfirst}
\usepackage{xargs}
\usepackage{ifthen}
\usepackage{fancyhdr}
\usepackage{latexsym}
\usepackage{lastpage}
\usepackage{textcomp}
\usepackage{varwidth}
\usepackage{caratula, aed2-tad,aed2-symb,aed2-itef}
\usepackage{algorithmicx, algpseudocode, algorithm}
\usepackage{enumerate}
\usepackage{graphicx}
\usepackage{caption}
\usepackage{subcaption}
\usepackage{float}
\usepackage{anysize}
\marginsize{1.5cm}{1.5cm}{1.5cm}{1.5cm}

\begin{document}

\titulo{Informe 3}
\materia{Algoritmos y Estructuras de Datos III}
\author{Grupo  \\Alvarez Vico Jazm\'in\\Cortés Conde Titó Javier María\\Pedraza Marcelo \\ Rozenberg Uriel Jonathan}

\integrante {Jazmín Alvarez Vico}{75/15}{jazminalvarezvico@gmail.com}
\integrante {Marcelo Pedraza}{393/14}{marcelopedraza314@gmail.com}
\integrante {Uriel Jonathan Rozenberg}{838/12}{rozenberguriel@gmail.com}
\integrante {Javier María Cortés Conde Titó}{252/15}{javiercortescondetito@gmail.com}

\maketitle


\clearpage

\tableofcontents
\cleardoublepage


\section{Introducción}
En nuestro problema Brian quiere convertirse en ``maestro pokemon'' en el menor tiempo posible. Para lograr este objetivo debe ir a todos los ``gimnasios'' y conquistarlos. Para poder hacerlo, cada gimnasio requiere una cantidad determinada de pociones. Estas pociones pueden obtenerse en las ``pokeparadas''. Las pokeparadas sólo pueden visitarse una vez y de cada visita obtenemos tres pociones.

Formalmente, podemos caracterizar nuestras pokeparadas y  gimnasios como nodos formando un grafo completo, es decir que existen aristas para unir cualquier par de nodo. nuestras aristas deben tener peso equivalente a la distancia entre dos nodos. Queremos encontrar el camino mínimo que une todos los nodos gimnasios y los nodos pokeparada que sean necesarios para poder conquistar todos los gimnasios. 

%%%%%%%%%%%%%%%%%%%%%%%%%%%%%%%%%%%%

\section{Algoritmo exacto}

\subsection{Explicación de la solución}

Para modelar este problema, creamos dos clases: Nodo y Mochila.
La clase Mochila contiene dos enteros ``capacidad'' el cual corresponde con la capacidad de la entrada (k) y ``peso'' en el cual se guarda la cantidad de pociones que se tiene en cada momento. 
\\
En el Pseudocódigo se llaman a las funciones DameCapacidad y DamePeso, las mismas devuelven estos valores.
\\
La clase Nodo tiene  un entero ``índice'' como identificador de cada nodoy un entero ``pociones'' en el cual se guardan las pociones que dan las pokeparadas o las que requieren los gimnasios(en este caso el valor es negativo). Además cuenta con un booleano para marcar si ya fue recorrido y dos enteros ``x'' e ``y'' para identificar las coordenadas del nodo.
\\
Nuestra solución utiliza la técnica de Backtracking para resolver el problema. Utilizamos variables globales (las cuales estan mencionadas en el pseudocodigo) ya que las necesitamos en todo momento en las funciones de nuestra resolución. De esta forma evitamos copiarlas cada vez, como parámetros de entrada.

Nuestro algoritmo de BT va formando recursivamente todas las soluciones posibles eligiendo gimnasios y pokeparas siempre que esto sea posible. Estas soluciones ``parciales'' son guardades en una variable que siempre es comparada con una variable de solución ``global''. En la misma se guarda siempre el mejor resultado obtenido hasta el momento. El ciclo principal de la función va desde 0 hasta el máximo entre el largo de los vectores de Nodos y pokeparadas. De esta forma vamos probando empezar con todos los nodos posibles. Cada vez que se elige un nodo para continuar se vuelve a llamar a la función BT para seguir construyendo la solución, luego se revierten los cambios para probar las demás soluciones. 

En nuestro algoritmo, cortamos una posible solución en el momento que supera la solución global encontrada hasta el momento. De esta forma evitamos seguir ejecutando el algoritmo para una solución que claramente no va a ser la mejor, y así reducimos el tiempo de cómputo.

Actualizamos la solución global, únicamente si logramos recorrer todos los gimnasios. Podemos saber esto ya que guardamos un entero contando los gimnasios recorridos. De esta forma si la solución queda vacia, sabemos que no existe solución para el problema y podemos devolver -1.

Mientras hacemos la lectura de datos, guardamos la cantidad máxima de pociones que requieren los gimnasios y el total de pociones que se requieren para ganar a todos los gimnasios. De esta forma si la capacidad de la mochila es menor al máximo de las pociones que piden los gimnasios o la cantidad de pociones que se pueden obtener (cantidad de pokeparadas multiplicado por tres) es menor a lo que se necesita para ganar a todos los gimnasios, entonces devolvemos -1 (evitando entrar en el backtracking) ya que el algoritmo no tiene solución. 



%falta cuando devuelve -1 salvo esos 2 casos siempre hay solucion?%

\subsection{Pseudocódigo}

\begin{algorithm}[H]{\textbf{variables Globales}}
	\begin{algorithmic}[1]
		\State real MinActual $\gets \infty$
		\State real MinGlobal $\gets 0$
		\State vint RecorridoGlobal
		\State vint RecorridoActual
		\State vnod PokeParadas
		\State vnod Gimnasios
		\State int PokeParadasRecorridas
		\State int PocionesNecesarias		
		\State Mochila Moch	
		\State entero GimRecorridos	
	
	\end{algorithmic}
\end{algorithm}

\begin{algorithm}[H]{\textbf{BT}()}
	\begin{algorithmic}[1]
		\State Si  MinActual$>$MinGlobal  
			 \State \quad cortar
		\State finSi
		\State Si GimRecorridos = $|$Gimnasios$|$ 
			\State \quad Si MinActual $<$ Min Gobal
				\State \qquad MinGlobal $\gets$ MinActual
				\State \qquad RecorridoGlobal $\gets$ RecorridoAtual
			\State \quad finSi
		\State finSi
		\State Desde i=0 hasta i$<$Max($|$Gimnasios$|$,$|$PokeParadas$|$)
			\State \quad Si i$< |$Gimnasios$|$
				\State \quad \quad gim $\gets$ Gimnasios[i]
				\State \quad \quad Si puedoIrGim(gim)
					\State \quad \quad \quad Voy(gim)
		 		\State \quad \quad finSi	
			\State \quad finSi
			\State \quad Si i $< |$PokeParadas$|$
				\State \qquad pp $\gets$ Pokeparadas[i]
				\State \qquad PokeParadasRecorridas $\gets$ PokeParadasRecorridas + 1 
				\State \qquad Si puedoIrPP(pp)
					\State \qquad \quad Voy(pp) 	
				\State \qquad finSi
				\State \qquad PokeParadasRecorridas $\gets$ PokeParadasRecorridas - 1 
			\State \quad finSi 
		\State finDesde
		\State Si RecorridoGlobal $= \emptyset$
			\State \quad Devolver -1
		\State Si no
			\State \quad Devolver MinGlobal
		\State finSi
 
	\end{algorithmic}
\end{algorithm}



\begin{algorithm}[H]{\textbf{Voy}(Nodo n)}
	\begin{algorithmic}[1]
		\State Marco n
		\State Si RecorridoActual $ \not= \emptyset$
		\State \quad origen $\gets$ ultimo de RecorridoActual
		\State \quad minActual $\gets$ minActual+ Distancia(origen, p)
		\State finSi
		\State Agrego n a RecorridoActual
		\State Modifico el peso de moch según las pociones n
		\State BT()
		\State revierto todas las modificaciones
	\end{algorithmic}
\end{algorithm}

Esta función actualiza los valores antes de hacer el llamado recursivo, y luego revierte las modificaciones locales(RecorridoActual, nodos recorridos, minActual).


\begin{algorithm}[H]{\textbf{puedoIrPP}(Nodo n){bool}}
	\begin{algorithmic}[1]
	\State NoConsumoDeMas $\gets$ Si al entrar a esta pokeparada se tiene que descartar pociones que no nos dejan ganar a todos los gimnasios asigno True, si no False.
	%terminar%
	\State  res $\gets$  moch no está llena y n no está recorrido y NoConsumoDeMas
	\end{algorithmic}
\end{algorithm}
Esto es una poda; calcula si esta pokeparada genera un estado sin solución trivial: las pokeparadas restantes y las pociones en la mochila no alcanzan para ganar los gimnasios que quedan.

\begin{algorithm}[H]{\textbf{PuedoIrGim}(Nodo n)}
	\begin{algorithmic}[1]
		\state res $\gets$  n no está recorrido y DamePeso(moch) $\geq$ -(DamePociones(n))
	\end{algorithmic}
\end{algorithm}

Simplemente analiza si lo puede vencer.

\subsection{Demostración de Correctitud}

Para mostrar la correctitud de nnuestro algoritmo queremos ver tre cosas: que el algoritmo termina, que analiza todos los casos posibles, y que deveulve la solución óptima.

\begin{itemize}
	\item El algoritmo termina:
Podemos ver que la cantidad de nodos en nuestro grafo es finita. Además mientras se está buscando la solución, no se puede recorrer dos veces el mismo nodo, entonces no puede recorrer dos veces la misma instancia ni desarrollarse algún ciclo infinito. Las llamadas recurisivas del algoritmo dependen de la cantidad de nodos del grafo. Puesto, que establecimos que esta cantidad era finita, podemos concluir que la cantidad de llamadas lo será también.
	\item El algoritmo analiza todos los casos:
Como explicamos en la sección 2.1, nuestro algoritmo itera sobre todos los nodos, probando empezar por cada uno de ellos siempre y cuando sea posible. Cada vez que se elige un nodo se vuelve a llamar al backtracking para continuar armando la solución y luego se revierten los cambios (se deselige) para poder probar otras soluciones eligiendo los demás nodos. Este proceso se realiza hasta que no queden más nodos para probar.
	\item El algoritmo devuelve  la solución óptima:
El algoritmo cuenta con una variable global que guarda la solución mínima encontrada hasta el momento. Cada vez que se encuentra otra posible solución, estas se comparan acutalizando la solución global de ser necesario. Como probamos que el algoritmo analiza todos los casos, podemos garatizar que devolvemos la solución óptima.    

\end{itemize}

\subsection{Complejidad y demostración}
Sea t la cantidad total de nodos, es decir t=n+m donde n es la cantidad de gimnasios y m la cantidad de pokeparadas.
Proponemos que la complejidad de nuestro algoritmo es t! y lo demostraremos a continuación:

Podemos visualizar a nuestro algoritmo como un árbol de soluciones. Podemos ver que en el peor caso, el camino recorre todos los nodos, tanto gimnasios como pokeparadas. 
para elegir el primer nodo realizamos un ciclo de t iteraciones. de ahi obtenemos el primer nivel del árbol que tiene t hojas. Ahora por cada una de esas hojas tenemos que elegir un nodo entre los t-1 restantes. 
asique nuestra complejidad asciende a t(t-1).
Así hasta llegar a elegir el ultimo nodo, esto nos costará \[
\prod_{i=0}^{t-1}t-i=t\cdot(t-1)\cdot(t-2)\cdot \ \ \cdot 1  = t!\] 

Entonces nuestra complejidad resulta $ \mathcal{O}$(t!)  

\subsection{Podas y Estrategias}
Para nuestro algoritmo implementamos tres podas y las nombramos con letras. Mantendremos esta notación más adelante en la sección de experimentación.
\begin{itemize}
	\item Poda A:
Esta poda se encuentra en la sección 2.2 en el alogoritmo BT en las lineas 1 a 3.
Basicamente, mientras se esta buscando una posible solución, nos permite cortar una rama en el momento inmediato que supera la solución mínima encontrada hasta el momento. 

	\item Poda B
Esta poda se encuentra en la sección 2.2 en el algoritmo puedoIrPP linea 2. 
Esta poda permite no recorrer pokeparadas una vez que la mochila esta llena, ya que recorrerlas no traería ningún beneficio ya que recorreríamos más distancia sin ningún beneficio, porque no podemos obtener pociones.

	\item Poda C
Esta poda se encuentra en la sección 2.2 en el algoritmo puedoIrPP linea 1. 
A diferencia de la poda B esta poda sirve para los casos en los que en la mochila quedan 1 o 2 lugares libres. Entonces Si al desperdiciar esas pociones, ya no se puede ganarle a los gimnasios que quedan por recorrer evita ir a pokeparadas.

\end{itemize}

Solo encontramos una estrategia:
Al iterar, recorremos primero las pokeparadas y después los gimnasios ya que consideramos que por lo general, es más probable que la solución se obtenga de empezar por una pokeparada.  

\subsection{Experimentación y análisis de resultados}

%%%%%%%%%%%%%%%%%%%%%%%%%%%%%%%%%%%%%


\section{Heurística constructiva golosa}

El metodo de heurística golosa consiste en elegir siempre la mejor opción posible para continuar y formar la solución total. En este caso el criterio para elegir una mejor opción sería la cercanía. Bus


\subsection{Explicación de la solución}
Vamos a dividir la solución en dos partes: por un lado, está la elección del primer nodo a visitar; por el otro, la estrategia para visitar todos los gimnasios.

\begin{itemize}
	\item Elegir el primero: como se explica en el pseudocódigo, primero buscamos empezar en gimnasios triviales; estos son gimnasios de fuerza 0, que no requieren ninguna poción. Si no encuentra ninguno, busca gimnasios ``fáciles''; que sólo requieran una pokeparada previa para vencerlos. En caso de existir alguno, empezamos en la pokeparada mas cercana (la estrategia es luego dentro del algoritmo dirigirse a este gimnasio fácil). Si nuevamente estamos sin suerte, empezamos desde una pokeparada cualquiera. En esta implementación seleccionaremos al primero de la lista.
	\item Una vez elegido el primero, siempre mira lo más cercano: primero analiza si puede ganar el gimnasio más cercano, de ser imposible, se dirige a la pokeparada más cercana. Una vez efectivizado el movimiento, itera hasta que no haya gimnasios para vencer, o no queden pokeparadas para recargar.
	
\end{itemize} 




\subsection{Pseudocódigo}

\begin{algorithm}[H]{\textbf{goloso}()}
	\begin{algorithmic}[1]
		\State distanciaRecorrida $\gets$ 0
		\State Recorrido $\gets$ $\emptyset$
		\State ElegirPrimero
		\State Mientras Gimnasios $\not= \emptyset$
		\State \quad Si puedo ganarle al gimnasio más cercano
		\State \qquad proxLugar=GimMasCercano
		\State \quad Si No quedan pokeparadas 
		\State \qquad distanciaRecorrida $\gets$ -1
		\State \qquad Recorrido $\gets$ $\emptyset$
		\State \qquad Cortar mientras
		\State \quad Si No
		\State \qquad proxLugar=PokeParadaMasCercana
		\State \quad finSi
		\State \quad moverse(proxLugar)
		\State finMientras
		\State Devolver distanciaRecorrida y Recorrido 
	\end{algorithmic}
\end{algorithm}

Esta es la función principal. Si puede vencer al gimnasio más cercano, lo hace. Si no, intenta ir a la pokeparada más cercana. Si ya no existen pokeparadas, devuelve -1.

\begin{algorithm}[H]{\textbf{moverse}(Nodo n)}
	\begin{algorithmic}[1]
		\State Modifico la mochila según las pociones de n
		\State Agregar atrás de Recorrido a n
		\State distanciaRecorrida + dist(n, posiciónActual)
		\State Actualizar posición actual
		\State Si n es gimnasio
		\State \quad sacar a n de Gimnasios
		\State Si No
		\State \quad sacar a n de pokeParadas
		\State finSi
	\end{algorithmic}
\end{algorithm}

Dado un nodo n este algoritmo actualiza las variables posición actual, distanciaRecorrida y Recorrido, y depende del tipo de nodo, lo elimina de la lista correspondiente.

\begin{algorithm}[H]{\textbf{ElegirPrimero}()}
	\begin{algorithmic}[1]
	\State Si existe un gimnasio que requiera requiera 0 pociones 
	\State \quad Agregar atrás el nodo a Recorrido
	\State Si no
	\State \quad Si existe un gimnasio que requiere menos de 4 pociones
	\State \qquad Elegir la pokeparada más cercana y agregarla a Recorrido
	\State \quad Si No 
	\State \qquad elegir cualquier pokeparada para empezar
	\State \quad finSi
	\State finSi
	\end{algorithmic}
\end{algotihm}

Este algoritmo es una manera de elegir el nodo inicial. Si existe un gimnasio trivial(necesita 0 pociones) lo elijo como nodo inicial, si no, busco un gimnasio que sólo necesite una visita a una pokeparada (3 pociones o menos) y elijo la pokeparada como nodo incial. Si no existe ninguno de estos, elijo una pokeparada al azar como nodo inicial.


\subsection{Complejidad}
Vamos a demostrar que la complejidad de esta heurística es $\mathcal{O}$($(n+m)^2$) donde n es la cantidad de gimnasios y m la cantidad de pokeparadas.
Por un lado, vamos a analizar las funciones auxiliares. La función $moverse$ usa todas operaciones en $\mathcal{O}$($1$) y no contiene ciclos. Entonces su complejidad es $\mathcal{O}$($1$).
Luego, ElegirPrimero recorre los gimnasios en busca de uno trivial ($\mathcal{O}$($n$)), luego recorre nuevamente los gimnasios en busca de uno ``fácil'', y por cada uno que encuentre busca la pokeparada más cercana ($\mathcal{O}$($n*m$)), y por último elige una pokeparada cualquiera ($\mathcal{O}$($1$)). En resumen, esta función auxiliar tiene complejidad $\mathcal{O}$($n*m$)
\\
Viendo el algoritmo principal vemos que cada iteración de la función toma dos posibilidades: o se mueve a un gimnasio o a una pokeparada. En el peor caso el algoritmo va a hacer $\mathcal{O}$($n+m$) iteraciones. Cada iteración primero calcula el gimnasio más cercano, y luego la pokeparada más cercana. Todas las demas operaciones son $\mathcal{O}$(1). En conclusión, cada iteración tiene complejidad  ($\mathcal{O}$($n+m$)).
Finalmente vemos que el algoritmo tiene complejidad $\mathcal{O}$($(n+m)^2$).


\subsection{Experimentación y análisis de resultados}


%%%%%%%%%%%%%%%%%%%%%%%%%%%%%%%%%%%%%


\section{Heurística de búsqueda local}

\subsection{Explicación de la solución}
Este algoritmo ordena los gimnasios por fuerza de menor a mayor, y luego ``rellena'' con las pokeparadas necesarias cada espacio entre gimnasios, de manera que en cada gimnasio se tienen las pociones necesarias para ser derrotado.
Luego, define su vecindad de soluciones como todas las soluciones que tengan un intercambio de pokeparadas. Supongamos que se tiene la solución \\
\\
 $P_{1}, P_{2} , G_{1}, P_{3}, P_{4}, P_{5}, G_{2}.$ \\
 \\
  Un vecino de esta solución es\\
  $P_{4}, P_{2} , G_{1}, P_{3}, P_{1}, P_{5}, G_{2}.$\\
  \\
  Intercambiando  $P_{1}$ con $P_{4}$\\
\\
Una vez calculados todos los vecinos posibles, se queda con el mínimo entre el conjunto de vecinos y la solución original. 
\subsection{Pseudocódigo}

%\quad
\begin{algorithm}[H]{\textbf{BusquedaLocal}()}
	\begin{algorithmic}[1]
	\State sol $\gets$ ordenar los gimnasios por fuerza, y poner en el medio las pokeparadas necesarias para vencerlo
	\State vecindad $\gets$ Vecindades(sol)
	\State res $\gets$ Min(todos los caminos de vecindad, el camino de sol)
	\State devolver res
	\end{algorithmic}
\end{algotihm}

\begin{algorithm}[H]{\textbf{Vecindades}(vnod sol)}
	\begin{algorithmic}[1]
	\State res $\gets$ 
	\State Desde i,j=0, hasta i,j= $|$pokeparadas$|$, mientras i $\neq$ j
		\State \quad vecino $\gets$ intercambiar en sol pokeparadas[i] con pokeparadas[j]
		\State \quad guardar vecino en res
	\State Fin Desde
	Devolver res
	\end{algorithmic}
\end{algotihm}

\subsection{Complejidad}

\subsection{Experimentación y análisis de resultados}

%%%%%%%%%%%%%%%%%%%%%%%%%%%%%%%%%%


\section{Metaheurística}

Como metaheurística decidimos implementar un algoritmo ``GRASP''
La técnica de este tipo de algoritmos consiste en combinar sucecivamente un algoritmo golozo randomizado para explorar nuevos espacios de soluciones y luego aplicar una busqueda local para mejorar cada solución. Al mismo tiempo se van comparando las soluciones y siempre se va eligiendo la mejor posible hasta que el algoritmo finaliza bajo algún criterio.

\subsection{Explicación de la solución}

Utilizamos como criterio de terminación el tamaño del conjunto de ``entradas validas''. Este conjunto esta formado por Todos los gimnasios que pidan cero pociones para ser vencidos y todas las pokeparadas. Aplicamos el algoritmo golozo comenzando de cada uno de los nodos de este conjunto. 

El algoritmo golozo randomizado es una variación de nuestra heuristica goloza presentada en la sección (). % insertar seccion%
Para explorar el nuevo espacio de soluciones seleccionamos como candidatos al veinte porciento de los nodos por recorrer más cercanos a la posición actual.
Luego le asignamos una probabilidad a cada uno de ellos. La misma utiliza el siguiente criterio:

\begin{itemize}
	\item Si el nodo es gimnasio y tengo suficientes pociones en mi mochila para ir, +30
	\item Si el nodo es pokeparada y entran las 3 pociones en la mochila, +25
	\item Si el nodo es pokeparada y entran 2 pociones en la mochila, +15
	\item Si el nodo es pokeparada y entra sólo 1 pocion en la mochila, +10
	\item Se le suma a cada nodo  bajo criterio de cercanía 10*$\# $MasCercanos - 10* i donde i representa la posición en ese orden.
\end{itemize} 

Guardamos el valor de la sumatoria de los valores asignados a cada nodo  y creamos un número aleatorio entre 1 y este valor. El nodo elegido será el nodo con valor mayor más proximo al número aleatorio. Estas acciones se realizan en la función que llamamos en el pseudocódigo ``ElegirProximo''. 

Finalmente aplicamos la función de busqueda local de la sección ().


\subsection{Pseudocódigo}

\begin{algorithm}[H]{\textbf{GRASP}(vnod entradasValidas)}
	\begin{algorithmic}[1]
		\State solu = $\infty$
		\State Desde i=0 hasta $|$entradasValidas$|$
		\State \quad GolozoRand(entradasValidas[i])
		\State \quad Blocas(RecorridoGlobal, MinGlobal)
		\State \quad Si MinGlobal$<$solu
		\State \qquad solu=MinGlobal
		\State \qquad RecorridoSolu=RecorridoGlobal
		\State \quad finSi
		\State fin Desde
		
	\end{algorithmic}
\end{algorithm}


\subsection{Complejidad}

\subsection{Experimentación y análisis de resultados}
\end{document}
