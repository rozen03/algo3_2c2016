\documentclass[spanish,12pt]{article}
\usepackage[spanish]{babel}
\usepackage[utf8]{inputenc}
\usepackage{xspace}
\usepackage{lmodern}
\usepackage{indentfirst}
\usepackage{xargs}
\usepackage{ifthen}
\usepackage{fancyhdr}
\usepackage{latexsym}
\usepackage{lastpage}
\usepackage{textcomp}
\usepackage{varwidth}
\usepackage{caratula, aed2-tad,aed2-symb,aed2-itef}
\usepackage{algorithmicx, algpseudocode, algorithm}
\usepackage{enumerate}
\usepackage{graphicx}
\usepackage{caption}
\usepackage{subcaption}
\usepackage{float}
\usepackage{anysize}
\marginsize{1.5cm}{1.5cm}{1.5cm}{1.5cm}

\begin{document}

\titulo{Informe 3}
\materia{Algoritmos y Estructuras de Datos III}
\author{Grupo  \\Alvarez Vico Jazm\'in\\Cortés Conde Titó Javier María\\Pedraza Marcelo \\ Rozenberg Uriel Jonathan}

\integrante {Jazmín Alvazer Vico}{75/15}{jazminalvarezvico@gmail.com}
\integrante {Marcelo Pedraza}{393/14}{marcelopedraza314@gmail.com}
\integrante {Uriel Jonathan Rozenberg}{838/12}{rozenberguriel@gmail.com}
\integrante {Javier María Cortés Conde Titó}{252/15}{javiercortescondetito@gmail.com}

\maketitle


\clearpage

\tableofcontents
\cleardoublepage


\section{Introducción}
En nuestro problema Brian quiere convertirse en "maestro pokemon" en el menor tiempo posible. Para lograr este objetivo debe ir a todos los "gimnasios" y conquistarlos. Para poder hacerlo, cada gimnasio requiere una cantidad determinada de pociones. Estas pociones pueden obtenerse en las "pokeparadas". Las pokeparadas solo pueden visitarse una vez y de cada visita obtenemos tres pociones.

Formalmente, podemos caracterizar nuestras pokeparadas y  gimnasios como nodos formando un grafo completo, es decir que existen aristas para unir cualquier par de nodo. nuestras aristas deben tener peso, que equivalga a la distancia entre dos nodos. entonces queremos encontrar el camino mínimo que une todo los nodos gimnasios y los nodos pokeparada que hagan falta para poder conquistar todos los gimnasios. 

%%%%%%%%%%%%%%%%%%%%%%%%%%%%%%%%%%%%

\section{Algoritmo exacto}

\subsection{Exlicación de la solución}

\subsection{Pseudocodigo}

\begin{algorithm}[H]{\textbf{Solución}(mochila moch, vnod  PokeParadas, vnod Gimnasios)}
	\begin{algorithmic}[1]
		\State MinActual $\gets \infty$
		\State MinGlobal $\gets 0$
		\State RecorridoGlobal $\gets$ vector de enteros
		\State RecorridoActual $\gets$ vector de enteros
		\State Si  MinActual$>$MinGlobal  
			 \State \quad cortar
		\State finSi
		\State Si GimRecorridos = $|$Gimnasios$|$ % ver donde declaro  GimRecorridos%
		\State \quad Si MinActual $<$ Min Gobal
		\State \qquad MinGlobal $\gets$ MinActual
		\State \qquad RecorridoGlobal $\gets$ RecorridoAtual
		\State \quad finSi
		\State finSi
		\State Desde i=0 hasta i$<$Max($|$Gimnasios$|$,$|$PokeParadas$|$)
		\State \quad Si RecorridoActual es vacio
		\State \qquad Si i$<|$PokeParadas$|$ 
		\State \qquad \quad Marco el nodo
		\State \qquad \quad agrego el indice del nodo a RecorridoActual
		\State \qquad \quad aumento el peso de la mochila a medida de que sea posible hasta tres
		\State \qquad \quad BT(mochila moch, vnod  PokeParadas, vnod Gimnasios)
		\State \qquad \quad revierto las modificaciones
		\State \qquad finSi
		\State \qquad Si
		\State \qquad \quad Si i$< |$Gimnasios$|$
		\State \qquad \qquad gim $\gets$ Gimnasios[i]
		\State \qquad \qquad Si puedoIr(gim)
		\State \qquad \qquad \quad GimRecorridos +1
		\State \qquad \qquad \quad Marco gim
		\State \qquad \qquad \quad Agrego el indice 
		\State \qquad \qquad fin Si
		\State \qquad \quad finSi
		\State \qquad finSi  			
		\State \quad finSi
		\State finDesde
  
	\end{algorithmic}
\end{algorithm}

\subsection{Complejidad}

\subsection{Experimentación y análisis de resultados}

%%%%%%%%%%%%%%%%%%%%%%%%%%%%%%%%%%%%%


\section{Heurística constructiva golosa}

\subsection{Exlicación de la solución}

\subsection{Pseudocodigo}

\subsection{Complejidad}

\subsection{Experimentación y análisis de resultados}


%%%%%%%%%%%%%%%%%%%%%%%%%%%%%%%%%%%%%


\section{Heurística de búsqueda local}

\subsection{Exlicación de la solución}

\subsection{Pseudocodigo}

\subsection{Complejidad}

\subsection{Experimentación y análisis de resultados}

%%%%%%%%%%%%%%%%%%%%%%%%%%%%%%%%%%


\section{Metaheurística}

\subsection{Exlicación de la solución}

\subsection{Pseudocodigo}

\subsection{Complejidad}

\subsection{Experimentación y análisis de resultados}
\end{document}
